\chapter{Employing GPUs in scientific algorithms}

The information age has brought a massive increase in the amount of data that is being collected and processed. The `data explosion' can be observed in virtually every field of computer science. In the scientific domain, this phenomenon is multiplied by increasingly powerful data gathering devices (i.e., cytometers in bioinformatics, optic sensors in physics, seismometers in geology, etc.). The amount of data is simply too large to be processed in a required time using. To alleviate this apparent pressure, the corresponding algorithm (which typically have super-linear time complexity) must be ported to massively parallel architectures, such as GPUs. The provided advantage can in return be the ability to assess bigger data, use more detailed processing methods or to analyze the data in real time.

This chapter enumerates four selected scientific algorithms and discusses their challenges in porting to GPUs to enable higher performance.

\section{Hierarchical clustering with the Ma\-ha\-la\-no\-bis linkage}

In the field of bioinformatics, the hierarchical clustering is a popular method for analyzing various types of data. 
The hierarchical clustering is an unsupervised machine learning method that aims to group the data points into clusters according to some linkage criterion. 
The simplest linkage may be computed as a minimum of Euclidean distances between the data points in the clusters.
But the Mahalanobis linkage has been shown to produce better results than the common linkages.

Mahalanobis linkage used Mahalanobis distance to measure the similarity of the data clusters. Mahalanobis distance is defined between a point $x$ and a set of points $P$ as
$$ d(x, P) = \sqrt{(x - P)^Tcov(P)(x - P)}$$,
where P is the set mean and cov(P) is the covariance matrix of the set P (if interpreted as a `distribution of points').
The distance has a specific property which can be intuitevely interpreted as a Euclidean distance to a center of a set which `accounts' the shape and size of the cluster. In other words, the Mahalanobis distance is a Euclidean distance in a transformed space, such that the covariance matrix of a cluster is the identity matrix.

In the Mahalanobis linkage, the distance between two clusters is defined as the average of Mahalanobis distances between their means and each other.
This allows for a very natural formation of clusters in biological data. However, the benefits come with the increase in complexity. 
To compute the Mahalanobis distance, we need to compute the mean of a cluster and an inverse of its covariance matrix and finally do an actual distance computation which are two matrix multiplications.

\subsection{Prior solution and its limitations}

High computational complexity limited the original implementation of the algorithm to a maximum thousands of dataset points. To overcome this limitation, the authors developed the method of apriori clusters to reduce the dataset size for the sake of the accuracy. The apriori clustering is a preprocessing step in which small clusters are created using Euclidean distance. Then these clusters are inputed to the main algorithm. This allowed to push the limits of the original implementation in orders of magnitude, but at a cost of less accurate results.

The memory complexity is also a limiting factor. The original implementation uses a dissimilarity matrix to store the distances between the clusters. Although this data structures saves computation time (in distance evaluation), it has quadratic memory complexity. This limits the size of the dataset to a few thousands of points.

\subsection{Optimized solution}

In our solution, we use a nearest neighbor array instead of the dissimilarity matrix. This allows us to reduce the memory complexity to linear. The nearest neighbor array is a one-dimensional array of size $n$ where $n$ is the number of clusters. Each element of the array contains the index of the nearest neighbor of the corresponding cluster. Although it requires more updates in the worst case, we used a simple optimization to buffer multiple nearest neighbors for each cluster.

The other challenge was that the complete solution was composed of multiple kernels of different duration. The computation of covariance matrix is fast at the beginning of the algorithm (because the clusters are small) while nearest neighbor list updates are slow (because there are more potential neighbors to update). Thankfully, these two kernels are independent and can be executed in parallel. We used CUDA streams to overlap the execution of these two kernels and thus keeping the high utilization of the GPU.

\section{Neighborhood-based dimensionality reduction}

definition of dimensionality reduction

limitations of real time processing


optimizations:

memory-heavy stuff:
- using registers as caches
- vector loads

\section{Cross-correlation}

much more detailed register usage 

\section{Stochastic simulation of Boolean networks}

definition of stochastic simulation

limitations of real time processing

runtime compilation and linkage of model code