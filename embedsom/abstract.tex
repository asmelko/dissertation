Dimensionality reduction methods have found vast applications as visualization tools in diverse areas of science.
Although many different methods exist, their performance is often insufficient for providing quick insight into many contemporary datasets.
In this paper, we propose a highly optimized GPU implementation of EmbedSOM, a dimensionality reduction algorithm based on self-organizing maps.
We detail the optimizations of $k$NN search and 2D projection kernels which comprise the core of the algorithm.
To tackle the thread divergence and low arithmetic intensity, we use a modified bitonic sort for $k$NN search and a projection kernel that utilizes vector loads and register caches.
The evaluated performance benchmarks indicate that the optimized EmbedSOM implementation is capable of projecting over 30 million individual data points per second.


% Dimensionality reduction methods have found vast application as visualization tools in diverse areas of science.
% Although many different methods exist, their performance is often insufficient for providing quick insight into many contemporary datasets, and the unsupervised mode of use prevents the users from utilizing the methods for dataset exploration and fine-tuning the details for improved visualization quality.
% We present BlosSOM, a high-performance semi-supervised dimensionality reduction software for interactive user-steerable visualization of high-dimensional datasets with millions of individual data points.
% BlosSOM builds on a GPU-accelerated implementation of the EmbedSOM algorithm, complemented by several landmark-based algorithms for interfacing the unsupervised model learning algorithms with the user supervision.
% We show the application of BlosSOM on realistic datasets, where it helps to produce high-quality visualizations that incorporate user-specified layout and focus on certain features.
% We believe the semi-supervised dimensionality reduction will improve the data visualization possibilities for science areas such as single-cell cytometry, and provide a fast and efficient base methodology for new directions in dataset exploration and annotation.
