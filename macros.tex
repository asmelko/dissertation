%%% This file contains definitions of various useful macros and environments %%%
%%% Please add more macros here instead of cluttering other files with them. %%%

%%% Minor tweaks of style

% This macro defines a chapter, which is not numbered, but is included
% in the table of contents.
\def\chapwithtoc#1{
\chapter*{#1}
\addcontentsline{toc}{chapter}{#1}
}

\def\secwithtoc#1{
\section*{#1}
\addcontentsline{toc}{section}{#1}
}

% Draw black "slugs" whenever a line overflows, so that we can spot it easily.
\overfullrule=5mm
\emergencystretch=1cm

\AtBeginDocument{
\hyphenation{SOM-Hun-ter}
\hyphenation{Wi-de-spread}
}

%%% Macros for definitions, theorems, claims, examples, ... (requires amsthm package)

\theoremstyle{plain}
\newtheorem{thm}{Theorem}
\newtheorem{lemma}[thm]{Lemma}
\newtheorem{claim}[thm]{Claim}

\theoremstyle{plain}
\newtheorem{defn}{Definition}
\newtheorem{problem}{Problem}

\theoremstyle{remark}
\newtheorem*{cor}{Corollary}
\newtheorem*{rem}{Remark}

%%% An environment for proofs

\newenvironment{myproof}{
  \par\medskip\noindent
  \textit{Proof}.
}{
\newline
\rightline{$\qedsymbol$}
}

%%% The field of all real and natural numbers
\newcommand{\R}{\mathbb{R}}
\newcommand{\N}{\mathbb{N}}
\newcommand{\Asymp}[1]{\ensuremath{\mathcal{O}\left(#1\right)}}

%%% Useful operators for statistics and probability
\DeclareMathOperator{\pr}{\textsf{P}}
\DeclareMathOperator{\E}{\textsf{E}\,}
\DeclareMathOperator{\var}{\textrm{var}}
\DeclareMathOperator{\sd}{\textrm{sd}}
\DeclareMathOperator{\subs}{\textsc{Sub}}

% contributions
\newcounter{contribution}
\def\contribution#1{
\refstepcounter{contribution}
\chapter*{Contribution \arabic{contribution}\\#1}
\addcontentsline{toc}{section}{Contribution \arabic{contribution}\\#1}
}

\crefname{contribution}{contribution}{contributions}
\Crefname{contribution}{Contribution}{Contributions}

\setlength{\marginparpush}{0pt}
\setlength{\marginparsep}{2em}

\newcommand{\contribmark}[1]{%
\hyperref[#1]{\begin{tikzpicture}[ultra thick, font=\sf\bfseries]
\node[regular polygon, regular polygon sides=3, rounded corners=0.3ex, inner sep=.3ex,draw] (ll) {\ref*{#1}};
\end{tikzpicture}}}
\newcommand{\margincontrib}[1]{\marginnote{\contribmark{#1}}}

\newcommand{\publishedas}[1]{\begin{description}\item[Published as]\begin{refsection}\fullcite{#1}\end{refsection}\end{description}}

% ONLINE VERSION without the manuscripts
\newif\ifonlineversion
\onlineversiontrue
\onlineversionfalse
\newcommand\onlineomit[1]{\ifonlineversion\vfill\begin{center}\fbox{\parbox{16.5em}{\footnotesize A re-printed manuscript that would appear here is omitted in the electronic version of the thesis.}}\end{center}\vfill\else#1\fi}

% publication contributions
\def\contF{\begin{tikzpicture}\fill circle(0.666ex);\end{tikzpicture}}
\def\contP{\begin{tikzpicture}\begin{scope}[rotate=90]\clip circle(0.666ex);\fill (-0.666ex,-0.666ex) rectangle(-0.1ex,0.666ex);\end{scope}\draw circle(0.666ex);\end{tikzpicture}}
\def\contN{\begin{tikzpicture}\draw[densely dotted] circle(0.666ex);\end{tikzpicture}}

% chemfig style
\renewcommand*\printatom[1]{\ensuremath{\mathsf{#1}}}
\setchemfig{atom sep=4ex, angle increment=30}
\catcode`\_=11
\tikzset{
	ddbond/.style args={#1}{
		draw=none,
		decoration={%
			markings,
			mark=at position 0 with {
				\coordinate (CF@startdeloc) at (0,\dimexpr#1\CF_doublesep/2)
				coordinate (CF@startaxis) at (0,\dimexpr-#1\CF_doublesep/2);
			},
			mark=at position 1 with {
				\coordinate (CF@enddeloc) at (0,\dimexpr#1\CF_doublesep/2)
				coordinate (CF@endaxis) at (0,\dimexpr-#1\CF_doublesep/2);
				\draw[dash pattern=on 2pt off 1.5pt] (CF@startdeloc)--(CF@enddeloc);
				\draw (CF@startaxis)--(CF@endaxis);
			}
		},
		postaction={decorate}
	}
}
\catcode`\_=8

% latexdiff preamble commands:
%DIF UNDERLINE PREAMBLE %DIF PREAMBLE
\RequirePackage[normalem]{ulem} %DIF PREAMBLE
\RequirePackage{color}\definecolor{RED}{rgb}{1,0,0}\definecolor{BLUE}{rgb}{0,0,1} %DIF PREAMBLE
\providecommand{\DIFadd}[1]{{\protect\color{green!50!black}\uwave{#1}}} %DIF PREAMBLE
\providecommand{\DIFdel}[1]{{\protect\color{red}\sout{#1}}}                      %DIF PREAMBLE
%DIF SAFE PREAMBLE %DIF PREAMBLE
\providecommand{\DIFaddbegin}{} %DIF PREAMBLE
\providecommand{\DIFaddend}{} %DIF PREAMBLE
\providecommand{\DIFdelbegin}{} %DIF PREAMBLE
\providecommand{\DIFdelend}{} %DIF PREAMBLE
\providecommand{\DIFmodbegin}{} %DIF PREAMBLE
\providecommand{\DIFmodend}{} %DIF PREAMBLE
%DIF FLOATSAFE PREAMBLE %DIF PREAMBLE
\providecommand{\DIFaddFL}[1]{\DIFadd{#1}} %DIF PREAMBLE
\providecommand{\DIFdelFL}[1]{\DIFdel{#1}} %DIF PREAMBLE
\providecommand{\DIFaddbeginFL}{} %DIF PREAMBLE
\providecommand{\DIFaddendFL}{} %DIF PREAMBLE
\providecommand{\DIFdelbeginFL}{} %DIF PREAMBLE
\providecommand{\DIFdelendFL}{} %DIF PREAMBLE


% Mahalanobis
\DeclareMathOperator{\cov}{\textbf{cov}}
\DeclareMathOperator{\mean}{mean}
\DeclareMathOperator{\argmin}{argmin}