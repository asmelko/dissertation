\chapwithtoc{Conclusion}

This thesis summarizes several efforts to improve the performance of complex and inefficient algorithms from contemporary scientific domains, improving their practical usability by applying high-performance computing principles. In the two chapters, it outlines the motivations and results of six research contributions, presented in \cref{chap:contributions}.

The first four contributions detail the parallelization and optimization challenges of the scientific algorithms which we have worked on during our research. The results include novel GPU applications that use novel data structures, promote high scalability, and utilize complicated GPU optimization techniques. The presented implementations provide orders of magnitude speedups over the prior state-of-the-art, enabling domain scientists to finish their analyses in minutes instead of days, process more data with greater accuracy, interactively visualize the results, and explore previously unattainable problem variations. The methodology of the works can serve as a useful support for other researchers in the domain of GPGPU computing.

The remaining contributions present the use cases of the novel HPC library Noarr, specializing in expressing the layout and traversal of $n$-dimensional arrays, which are the most commonly used data structures in scientific computing. The novel approach of assigning names to the dimensions of the arrays and expressing the layout and traversal in a declarative way allows users to deploy memory-related optimizations in a more readable and maintainable manner, while the library takes care of the complex indexing and loop transformations.

As mentioned throughout the thesis, use of GPUs for general computing in becoming increasingly popular. Consequently, they are becoming increasingly more versatile with each new addition of new core types, specialized high-throughput instructions and new thread hierarchies. This creates new opportunities of an interesting future research: The implementation of Mahalanobis Hierarchical Clustering can be directly extended with the use of tensor cores, which may provide an additional order of magnitude performance improvement thanks to their high compute bandwidth. Further, we are planning to continue on $k$-NN development, which we have already started during the work on EmbedSOM; we believe that increasing the cache size by employing the new feature of distributed shared memory may improve the data throughput of such highly memory-bound algorithm. We would like to finish our work on biological simulations, especially BioFVM and PhysiCell, which has partly guided Noarr development due to the presence of multiple complex high-dimensional arrays.

Finally we hope that the contributions of this thesis will help to address the challenges posed by the increasing complexity of GPUs and the widening gap between compute and memory bandwidth. We believe that the presented methodologies and contributions provide valuable insights and can serve as a foundation for further research in the field of GPGPU computing and HPC.
