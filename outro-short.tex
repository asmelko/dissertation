\chapwithtoc{Conclusion}

This thesis summarizes several efforts to improve the performance of complex algorithms with inefficient implementations from contemporary scientific domains, improving their practical usability by applying high-performance computing principles. In the two chapters, it outlines the motivations and results of six research contributions.

The first four contributions detail the parallelization and optimization challenges of the scientific algorithms which we have worked on during our research. The results include GPU applications that use novel data structures, promote high scalability, and utilize complicated GPU optimization techniques. The presented implementations provide orders of magnitude speedups over the prior state-of-the-art, enabling domain scientists to finish their analyses in minutes instead of days, process more data with greater accuracy, interactively visualize the results, and explore previously unattainable problem variations. The significant speedups not only enhance the productivity of scientists but also decrease the energy consumption of computational tasks, making the research more sustainable. Furthermore, by bringing high-performance computation to GPU-equipped personal computers, we democratize access to powerful computational tools, allowing a broader range of scientists to perform advanced analyses without the need for expensive supercomputing resources.

The last two contributions present the use cases of the novel HPC library Noarr, specializing in expressing the layout and traversal of $n$-dimensional arrays, which are the most commonly used data structures in scientific computing. The novel approach of assigning names to the dimensions of the arrays and expressing the layout and traversal in a declarative way allows users to deploy memory-related optimizations in a more readable and maintainable manner while the library takes care of the complex indexing and loop transformations.

Finally, we hope that the contributions of this thesis will help to develop software that performs well in spite of the impending technical challenges and progressive widening of the bandwidth-compute gap. We believe that the presented methodologies and contributions provide valuable insights that can serve as a foundation for further research in the field of GPGPU computing and HPC.
