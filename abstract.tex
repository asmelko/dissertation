%%% A template for a simple PDF/A file like a stand-alone abstract of the thesis.

\documentclass[12pt]{report}

\usepackage[a4paper, hmargin=1in, vmargin=1in]{geometry}
\usepackage[a-2u]{pdfx}
\usepackage[utf8]{inputenc}
\usepackage[T1]{fontenc}
\usepackage{lmodern}
\usepackage{textcomp}

\begin{document}

This thesis studies, develops and investigates the optimization of data-intensive scientific algorithms using Graphics Processing Units (GPUs) to enhance performance and scalability. The first part of the thesis focuses on the design and implementation of optimized kernels for four key algorithms: hierarchical clustering with Mahalanobis linkage, neighborhood-based dimensionality reduction through EmbedSOM, optimization of cross-correlation algorithms for many small inputs, and stochastic simulation of Boolean networks. In the second part, the thesis builds upon the findings of the first part to propose a Noarr library, which enables the efficient development of high-performance computing (HPC) applications. In emphases the critical role of memory optimization in achieving significant performance improvements in HPC and aims to streamline the implementation of these optimizations by providing novel memory layout and traversal optimization framework. The main contributions of this thesis comprise of implementation of novel GPU optimization techniques, performance improvements of scientific tools of up to three orders of magnitude speedup, advancing data analysis and visualization in bioinformatics and material physics, and the design of new tools for efficient expression of data structure layout and traversal in HPC code. The results of this thesis may be used to enhance the development process of maintainable and efficient HPC applications, and guide future research in the field of data-intensive scientific computing.

\end{document}
